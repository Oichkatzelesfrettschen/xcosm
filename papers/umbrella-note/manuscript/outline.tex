% outline.tex
% COSMOS Program Overview - LaTeX Outline
% Short 6-10 page overview of research PROGRAM (not unified theory)
%
% CRITICAL CONSTRAINTS:
% - Must NOT include BIC comparisons or "tension resolved" language
% - Reference individual papers for details
% - Explicit "What We Do NOT Claim" section
%
% To be written AFTER individual papers are complete

\documentclass[12pt, letterpaper]{article}

\usepackage[utf8]{inputenc}
\usepackage[margin=1in]{geometry}
\usepackage{amsmath, amssymb, amsthm}
\usepackage{graphicx}
\usepackage{hyperref}
\usepackage{longtable}
\usepackage{booktabs}
\usepackage{xcolor}

\title{COSMOS: A Falsification Program Linking Exceptional Algebra Motifs, \\
       Discrete Cosmogenesis, and Supernova Systematics}

\author{%
  [Author Name]\\
  \small [Institution]\\
  \small \texttt{email@institution.edu}
}

\date{\today}

\begin{document}

\maketitle

\begin{abstract}
\noindent
COSMOS is a \textbf{research program}---not a unified theory---exploring three loosely connected hypothesis clusters: (1) exceptional algebraic structures (octonions, Jordan algebras, $F_4/E_6$) as organizing motifs for cosmological parameter values, (2) discrete pregeometry via bigraph dynamics with Ollivier-Ricci curvature, and (3) Type Ia supernova population evolution as a systematic affecting dark energy inference.

We present the dependency structure of hypotheses, distinguishing genuinely independent inputs from calibrated parameters, and enumerate falsification tests with explicit timelines. We report three falsified hypotheses ($\varphi^{-n}$ fermion masses, bigraph OR convergence, Strong CP original formulation) and five pending tests (Spandrel host bias, $H_0(R)$ scale-dependence, $r = 0.0048$ tensor modes, $\delta_{CP} = 67.8^\circ$, connected-regime OR convergence).

The appearance of $\varepsilon = 1/4$ in multiple contexts is used as an \textit{organizing motif} for hypothesis generation, \textbf{not} as evidence of deep unification. Of six claimed appearances, only two (black hole entropy, percolation threshold) are genuinely independent of cosmological calibration targets.

This note does NOT claim to resolve cosmological tensions, demonstrate emergent general relativity, or provide statistically significant detections. It invites critical evaluation of a falsifiable research program organized around algebraic and discrete-geometric constraints.

\textbf{Keywords:} cosmology, dark energy, Type Ia supernovae, exceptional algebras, discrete geometry, falsification
\end{abstract}

\section{Introduction}
\label{sec:intro}

% 1 page: Three loosely connected research directions
% Explicit framing: "research program with testable hypotheses"
% NOT: "unified framework that resolves tensions"

COSMOS is a research program, not a unified theory. It explores connections between three domains:

\subsection{Three Research Directions}

\paragraph{Direction 1: Algebraic Motifs}
Exceptional algebraic structures (octonions $\mathbb{O}$, Jordan algebra $J_3(\mathbb{O})$, Lie groups $F_4, E_6, G_2$) produce numerical relationships that approximately match cosmological observables when one adopts the organizing principle $\varepsilon = 1/4$.

\textbf{Status:} Most appearances are calibrated to data, not predictions. See Section~\ref{sec:epsilon_motif}.

\paragraph{Direction 2: Discrete Cosmogenesis}
Bigraph rewriting dynamics with Ollivier-Ricci (OR) curvature flow provide a toy model for emergent spacetime geometry. Under this model, curvature should converge to general relativistic Ricci curvature in the continuum limit.

\textbf{Status:} Current simulations show \textit{divergence}, not convergence ($\kappa_{OR} \sim -N^{0.55}$). See Papers 3 and Section~\ref{sec:falsified}.

\paragraph{Direction 3: Supernova Systematics}
The ``Spandrel hypothesis'' proposes that DESI DR2's preference for evolving dark energy ($w_0 > -1, w_a < 0$ at 2.8--4.2$\sigma$) arises from unmodeled Type Ia supernova population evolution correlated with host galaxy properties (mass, metallicity, star formation).

\textbf{Status:} Primary falsification test pending (Paper 1). See Section~\ref{sec:pending_tests}.

\subsection{What This Program Is}

\begin{itemize}
  \item A collection of \textbf{testable hypotheses} with explicit falsification conditions
  \item A \textbf{dependency analysis} distinguishing independent inputs from calibrated fits
  \item An \textbf{invitation for critical evaluation} through reproducible data analysis
\end{itemize}

\subsection{What This Program Is NOT}

\begin{itemize}
  \item \textbf{NOT} a unified field theory
  \item \textbf{NOT} a resolution of cosmological tensions (Hubble, $S_8$, etc.)
  \item \textbf{NOT} a demonstration of emergent general relativity (convergence not shown)
  \item \textbf{NOT} a statistically significant detection (previous ``4.7$\sigma$'' H$_0$ claims suspended)
  \item \textbf{NOT} a claim of ``multiple independent confirmations'' (most $\varepsilon = 1/4$ appearances are calibrations)
\end{itemize}

\section{Hypothesis Inventory}
\label{sec:hypotheses}

% 2 pages: Detailed hypothesis table with dependency structure

We enumerate all COSMOS hypotheses, classified by independence structure. Table~\ref{tab:hypothesis_inventory} provides the full inventory.

\subsection{Classification System}

\paragraph{Independent Inputs (I):} Established results from independent domains (e.g., black hole thermodynamics, lattice theory, division algebras). These are \textit{not} COSMOS predictions.

\paragraph{Calibrated Parameters (C):} Quantities explicitly fitted to observational data. These are \textbf{not} predictions and provide \textbf{no evidential support} for the program.

\paragraph{Derived Hypotheses (D):} Theoretical constructions built from independent inputs and/or calibrated parameters.

\paragraph{Testable Predictions (P):} Falsifiable claims about future observations.

\begin{table}[h]
\centering
\caption{COSMOS Hypothesis Inventory}
\label{tab:hypothesis_inventory}
\begin{tabular}{llcp{5cm}}
\toprule
ID & Hypothesis & Testable? & Description \\
\midrule
\multicolumn{4}{c}{\textit{Independent Inputs}} \\
\midrule
I1 & BH entropy $S = A/4$ & No & Bekenstein-Hawking formula (Planck units) \\
I2 & Percolation $p_c = 1/4$ & No & Bond percolation threshold (specific lattices) \\
I3 & Octonion structure & No & Division algebra $\mathbb{O}$ structure constants \\
I4 & Jordan algebra $J_3(\mathbb{O})$ & No & $3 \times 3$ Hermitian octonion matrices \\
\midrule
\multicolumn{4}{c}{\textit{Calibrated Parameters (NOT Predictions)}} \\
\midrule
C1 & $w_0 = -5/6$ & No & Dark energy EOS calibrated to DESI DR2 \\
C2 & $n_s = 0.966$ & No & Scalar spectral index calibrated to Planck \\
C3 & $S_8 = 0.78$ & No & Clustering amplitude calibrated to weak lensing \\
C4 & $H_0$ gradient slope & No & Scale-dependent expansion calibrated to distance ladder \\
\midrule
\multicolumn{4}{c}{\textit{Derived Hypotheses}} \\
\midrule
D1 & $\varepsilon = 1/4$ motif & No & Organizing principle (see Section~\ref{sec:epsilon_motif}) \\
D2 & $F_4 \to J_3(\mathbb{O})$ & No & Exceptional algebra connection \\
D3 & Bigraph rewriting & Yes & Discrete pregeometry dynamics (Paper 3) \\
\midrule
\multicolumn{4}{c}{\textit{Testable Predictions}} \\
\midrule
P1 & Spandrel host bias & Yes & SN Ia systematics (Paper 1, 2025 Q1--Q2) \\
P2 & $r = 0.0048$ & Yes & Tensor modes (CMB-S4, 2029--2032) \\
P3 & $\delta_{CP} = 67.8^\circ$ & Yes & CKM phase (LHCb/Belle II, 2025--2028) \\
P4 & OR $\to$ Ricci convergence & Yes & Bigraph curvature (Paper 3, \textbf{FALSIFIED}) \\
P5 & $H_0(R)$ trend & Yes & Scale-dependent expansion (Paper 2, 2025--2026) \\
\bottomrule
\end{tabular}
\end{table}

\subsection{Critical Observation: Most $\varepsilon = 1/4$ Appearances Are Calibrated}

Of the six claimed appearances of $\varepsilon = 1/4$:
\begin{itemize}
  \item \textbf{2 independent} (I1: BH entropy, I2: percolation)
  \item \textbf{3 calibrated} (C1: $w_0$, C2: $n_s$, C3: $S_8$)
  \item \textbf{1 theoretical} (Jordan algebra structure, unproven connection)
\end{itemize}

This is \textbf{not} ``convergent evidence''---it is an \textit{organizing motif} used to generate hypotheses. See Section~\ref{sec:epsilon_motif} for detailed analysis.

\section{Dependency Graph}
\label{sec:dependencies}

% 1 page: Visual dependency structure + brief discussion

Figure~\ref{fig:dependency_graph} shows the hypothesis dependency graph. Arrows indicate logical/causal dependencies. Red nodes (calibrated parameters) are \textbf{fitted to data} and provide \textbf{no independent confirmation}.

\begin{figure}[h]
\centering
% Include generated figure from hypothesis_matrix.py
\includegraphics[width=0.9\textwidth]{figures/hypothesis_dependency.pdf}
\caption{COSMOS hypothesis dependency graph. \textcolor{red}{Red nodes} are calibrated (not predictions). \textcolor{green}{Green nodes} are independent inputs. \textcolor{orange}{Orange nodes} are testable predictions. Most dependencies flow through calibrated parameters, not independent inputs.}
\label{fig:dependency_graph}
\end{figure}

\subsection{Interpretation}

The dependency graph reveals:
\begin{enumerate}
  \item Only 4 genuinely independent inputs (I1--I4)
  \item All testable predictions depend heavily on calibrated parameters
  \item The $\varepsilon = 1/4$ motif (D1) receives input from both independent (I1, I2) and calibrated (C1, C2, C3) sources
\end{enumerate}

This structure undermines claims of ``multiple independent confirmations.''

\section{Falsification Program}
\label{sec:falsification}

% 2 pages: Falsified + pending tests with explicit conditions

\subsection{Falsified Hypotheses}
\label{sec:falsified}

COSMOS has generated and tested multiple hypotheses. Three have been definitively falsified (Table~\ref{tab:falsified}).

% Include table from falsification_table.py
% Falsified Hypotheses Table
\begin{table}[h]
\centering
\caption{Falsified COSMOS Hypotheses}
\label{tab:falsified}
\begin{tabular}{lllp{4cm}}
\toprule
ID & Hypothesis & Result & Evidence \\
\midrule
H1 & $\varphi^{-n}$ fermion masses & Falsified & $\chi^2/\text{dof} = 2931$ \\
H2 & OR curvature convergence & Falsified & $\kappa_{OR} \sim -N^{0.55}$ (divergence) \\
H3 & Strong CP $\theta = \pi/8$ & Falsified & $\theta < 10^{-10}$ (nEDM limit) \\
\bottomrule
\end{tabular}
\end{table}


\paragraph{H1: $\varphi^{-n}$ Fermion Mass Scaling}
Early exploration tested whether fermion masses follow $m_f = m_0 \varphi^{-n}$ (golden ratio scaling). Result: $\chi^2/\text{dof} = 35,173 / 12 = 2,931$. \textbf{Catastrophically falsified.}

\paragraph{H2: Bigraph OR Curvature Convergence}
The central hypothesis of discrete cosmogenesis is that Ollivier-Ricci curvature $\kappa_{OR}$ should converge to Ricci curvature $R_{\mu\nu}$ in the continuum limit. Simulations show \textit{divergence}: $\kappa_{OR} \sim -N^{0.55}$.

\textbf{Diagnosis:} CCF uses disconnected bigraphs, operating outside the regime where van der Hoorn et al.\ (2021, 2023) proved convergence (which requires connected graphs). This is an \textbf{open problem}, not a demonstrated result. See Paper 3.

\paragraph{H3: Strong CP (Original Formulation)}
Original hypothesis: $\theta_{QCD} = \pi/8$ from octonion geometry. Neutron electric dipole moment limits require $\theta < 10^{-10}$, but $\pi/8 \approx 0.393$. \textbf{Immediately falsified.} (Could be reframed after axion solution, but not pursued.)

\subsection{Pending Falsification Tests}
\label{sec:pending_tests}

Five testable predictions remain under evaluation (Table~\ref{tab:pending_tests}).

% Include table from falsification_table.py
% Pending Falsification Tests Table
\begin{table}[h]
\centering
\caption{Pending Falsification Tests}
\label{tab:pending_tests}
\begin{tabular}{llll}
\toprule
ID & Prediction & Test & Timeline \\
\midrule
P1 & Spandrel host bias & SN Ia sample split & 2025 Q1--Q2 \\
P2 & $H_0(R)$ trend & Multi-scale analysis & 2025--2026 \\
P3 & $r = 0.0048$ & CMB-S4 B-modes & 2029--2032 \\
P4 & $\delta_{CP} = 67.8^\circ$ & LHCb/Belle II & 2025--2028 \\
P5 & Connected OR convergence & Bigraph simulation & 2025 Q2 \\
\bottomrule
\end{tabular}
\end{table}


\paragraph{P1: Spandrel Host Galaxy Bias (HIGHEST PRIORITY)}
\textbf{Test:} Split Pantheon+/DES-SN sample by host mass, metallicity proxy, or specific star formation rate. Check for systematic bias in Hubble residuals correlated with redshift.

\textbf{Timeline:} 2025 Q1--Q2

\textbf{Success condition:} Bias detected, Hubble residuals shift with host properties, $\Delta(w_0, w_a) > 1\sigma$ after correction.

\textbf{Failure condition:} No bias pattern, or $\Delta(w_0, w_a) < 0.5\sigma$.

\textbf{Status:} Model development phase (Paper 1).

\paragraph{P2: $H_0(R)$ Scale-Dependence}
\textbf{Test:} Define local expansion rate $H_0(R)$ as a function of smoothing scale $R$ (Mpc). Compare trend to $\Lambda$CDM cosmic variance predictions from mock catalogs.

\textbf{Timeline:} 2025--2026

\textbf{Previous claim:} ``4.7$\sigma$ detection'' (SUSPENDED---heuristic $k$-mapping, not physically defined $R$).

\textbf{Current status:} Methodology development (Paper 2).

\paragraph{P3: Tensor-to-Scalar Ratio $r = 0.0048$}
\textbf{Test:} CMB B-mode polarization detection by CMB-S4 or LiteBIRD.

\textbf{Timeline:} 2029--2032

\textbf{Failure condition:} $r < 0.003$ (2$\sigma$ below) or $r > 0.007$ (2$\sigma$ above).

\textbf{Current limit:} $r < 0.032$ (Planck + BICEP/Keck).

\paragraph{P4: CP Phase $\delta_{CP} = 67.8^\circ$}
\textbf{Test:} Precision measurement of CKM unitarity triangle angle $\gamma$ (related to $\delta_{CP}$ via $\gamma \approx 60^\circ + \delta_{CP}$).

\textbf{Current data:} LHCb 2025 reports $\gamma = 62.8^\circ \pm 2.6^\circ$, predicting $\gamma \approx 67.8^\circ$.

\textbf{Tension:} 1.9$\sigma$ (marginal).

\textbf{Failure threshold:} $> 3\sigma$ deviation.

\textbf{Timeline:} 2025--2028 (improved LHCb/Belle II measurements).

\paragraph{P5: OR Convergence in Connected Regime}
\textbf{Test:} Modify CCF bigraph rewriting to maintain graph connectivity. Retest OR curvature convergence.

\textbf{Hypothesis:} Disconnected graphs cause divergence.

\textbf{Failure condition:} Divergence persists even in connected regime.

\textbf{Timeline:} 2025 Q2

\textbf{Status:} Under analysis (Paper 3).

\section{The $\varepsilon = 1/4$ Motif: Calibration vs.\ Prediction}
\label{sec:epsilon_motif}

% 1 page: Explicit breakdown of what is calibrated vs independent

\subsection{Six Claimed Appearances}

The organizing principle of COSMOS is that $\varepsilon = 1/4$ appears in multiple domains:

\begin{enumerate}
  \item \textbf{Black hole entropy:} $S = A / 4$ (Bekenstein-Hawking)
  \item \textbf{Percolation threshold:} $p_c = 1/4$ (specific lattices)
  \item \textbf{Dark energy EOS:} $w_0 = -(1 + \varepsilon) = -5/6$
  \item \textbf{Scalar spectral index:} $n_s = 1 - 4\varepsilon = 0.966$
  \item \textbf{Matter clustering:} $S_8 = 2 \sqrt{\varepsilon} \approx 0.78$
  \item \textbf{Jordan algebra:} Dimension 27 = $3^3$ connects to $F_4$ (78 generators)
\end{enumerate}

\subsection{Critical Analysis: Only 2 Are Independent}

\begin{table}[h]
\centering
\caption{$\varepsilon = 1/4$ Appearance Classification}
\label{tab:epsilon_classification}
\begin{tabular}{lll}
\toprule
Appearance & Source & Classification \\
\midrule
BH entropy & Bekenstein-Hawking (1973) & \textcolor{green}{Independent} \\
Percolation & Lattice theory (proven) & \textcolor{green}{Independent} \\
$w_0 = -5/6$ & DESI DR2 fit & \textcolor{red}{Calibrated} \\
$n_s = 0.966$ & Planck fit & \textcolor{red}{Calibrated} \\
$S_8 = 0.78$ & Weak lensing fit & \textcolor{red}{Calibrated} \\
Jordan algebra & Theoretical connection & \textcolor{blue}{Unproven} \\
\bottomrule
\end{tabular}
\end{table}

\subsection{Conclusion}

The appearance of $\varepsilon = 1/4$ is an \textbf{organizing motif} used to generate hypotheses, \textbf{not} evidence of deep unification. Of six appearances, only two are genuinely independent of the cosmological data being ``explained.''

This does \textbf{not} invalidate the research program---it clarifies the epistemic status of the hypotheses.

\section{Individual Papers}
\label{sec:papers}

% Brief (1/2 page each) summary of the three main papers

\subsection{Paper 1: Spandrel (Supernova Systematics)}

\textbf{Title:} ``Does SN Ia Progenitor Evolution Mimic Evolving Dark Energy?''

\textbf{Hypothesis:} DESI DR2's preference for $w_0 > -1, w_a < 0$ arises from unmodeled SN Ia population evolution correlated with host galaxy metallicity, mass, and star formation rate.

\textbf{Method:} Hierarchical Bayesian model splitting SN sample by host properties. Quantify $\Delta(w_0, w_a)$ after systematic correction.

\textbf{Status:} Model development (2025 Q1--Q2).

\subsection{Paper 2: Scale-Dependent $H_0$}

\textbf{Title:} ``Local Expansion Rate as a Function of Smoothing Scale''

\textbf{Hypothesis:} Local $H_0$ varies with smoothing scale $R$ (Mpc) due to cosmic variance and/or modified gravity.

\textbf{Method:} Define $H_0(R)$ estimator using multi-method distance ladder. Compare to $\Lambda$CDM mock catalogs.

\textbf{Previous claim:} ``4.7$\sigma$ detection'' (SUSPENDED---heuristic $k$-mapping).

\textbf{Status:} Methodology development (2025--2026).

\subsection{Paper 3: CCF Ollivier-Ricci Curvature}

\textbf{Title:} ``Ollivier-Ricci Curvature Under Bigraph Rewriting: Convergence Failure and Diagnostic Analysis''

\textbf{Hypothesis:} OR curvature should converge to Ricci curvature in continuum limit.

\textbf{Result:} \textbf{Divergence} ($\kappa_{OR} \sim -N^{0.55}$).

\textbf{Diagnosis:} CCF operates outside proven convergence regime (disconnected graphs).

\textbf{Status:} Failure mode analysis (2025 Q2).

\section{What We Do NOT Claim}
\label{sec:disclaimers}

% Explicit enumeration of overclaims to avoid

To avoid misinterpretation, we explicitly state what COSMOS does \textbf{NOT} claim:

\subsection{Not Claimed: Tension Resolution}

\begin{itemize}
  \item \textbf{NOT} claiming to resolve Hubble tension ($H_0$ discrepancy between early/late universe)
  \item \textbf{NOT} claiming to resolve $S_8$ tension (CMB vs.\ weak lensing)
  \item \textbf{NOT} claiming to explain DESI dark energy signal (pending Spandrel test)
\end{itemize}

\subsection{Not Claimed: Emergent General Relativity}

\begin{itemize}
  \item \textbf{NOT} claiming to have demonstrated GR emergence from bigraphs
  \item Current result: \textbf{divergence}, not convergence
  \item This is an \textbf{open problem} requiring resolution
\end{itemize}

\subsection{Not Claimed: Statistical Significance}

\begin{itemize}
  \item Previous ``4.7$\sigma$ detection'' of scale-dependent $H_0$ is \textbf{SUSPENDED}
  \item Heuristic $k$-mapping does not define physically interpretable scale $R$
  \item Significance cannot be assessed without $\Lambda$CDM null comparison
\end{itemize}

\subsection{Not Claimed: Multiple Independent Confirmations}

\begin{itemize}
  \item $\varepsilon = 1/4$ appears in 6 contexts
  \item Only 2 are independent (BH entropy, percolation)
  \item 3 are calibrated to cosmological data (not predictions)
  \item This is an organizing motif, not convergent evidence
\end{itemize}

\subsection{Not Claimed: Bayesian Model Preference}

\begin{itemize}
  \item \textbf{NOT} claiming ``BIC prefers COSMOS over $\Lambda$CDM''
  \item No reproducibility package released
  \item Previous BIC comparisons cannot be verified
  \item Avoid model selection claims until likelihood code is public
\end{itemize}

\section{Conclusion}
\label{sec:conclusion}

% 1 page: Invitation for critical evaluation, limitations, emphasis on falsifiability

COSMOS is a \textbf{falsification program}, not a unified theory. We have enumerated:

\begin{itemize}
  \item 4 independent inputs
  \item 4 calibrated parameters (not predictions)
  \item 3 derived hypotheses
  \item 5 testable predictions
\end{itemize}

\subsection{Current Scorecard}

\begin{itemize}
  \item \textbf{Falsified:} 3 hypotheses (fermion masses, OR convergence, Strong CP)
  \item \textbf{Pending:} 5 tests (Spandrel, $H_0(R)$, $r$, $\delta_{CP}$, connected OR)
  \item \textbf{Marginal:} 1 test ($\delta_{CP}$ at 1.9$\sigma$)
\end{itemize}

\subsection{Invitation for Critical Evaluation}

We invite the community to:
\begin{enumerate}
  \item Test the Spandrel hypothesis on SN data (Paper 1 analysis pending)
  \item Challenge the $H_0(R)$ methodology (Paper 2 under development)
  \item Propose alternative explanations for OR divergence (Paper 3)
  \item Identify additional falsification tests
\end{enumerate}

\subsection{Honest Limitations}

\begin{enumerate}
  \item Bigraph simulations operate outside proven convergence regime
  \item $H_0(k)$ analysis used heuristic mapping; significance claims suspended
  \item Most $\varepsilon = 1/4$ appearances are calibration targets, not independent confirmations
  \item Spandrel model is illustrative; not yet validated on real SN data with proper selection
\end{enumerate}

\subsection{Final Statement}

COSMOS prioritizes \textbf{falsifiability over confirmation}. We report failures prominently and suspend overclaims. The program stands or falls on the outcome of pending tests, particularly the Spandrel host galaxy bias analysis (2025 Q1--Q2).

If you find errors, identify overclaims, or propose sharper falsification tests, we welcome your input.

\section*{Acknowledgments}

[To be added after individual papers are complete.]

\bibliographystyle{plain}
\bibliography{cosmos_program}

% Bibliography will include:
% - DESI Collaboration (2024, 2025)
% - van der Hoorn et al. (2021, 2023) [OR curvature]
% - Scolnic et al. (2022) [Pantheon+]
% - LHCb Collaboration (2025) [gamma measurement]
% - Planck Collaboration (2020)
% - BICEP/Keck Collaboration
% - Rigault et al. (2020) [SN host effects]

\appendix

\section{Hypothesis Dependency Matrix}
\label{app:dependency_matrix}

[Include full adjacency matrix from hypothesis\_matrix.py]

\section{Complete Falsification Timeline}
\label{app:timeline}

[Include Gantt chart or timeline visualization of all tests]

\section{Reproducibility}

All analysis code is available at:
\begin{verbatim}
https://github.com/[username]/cosmos
\end{verbatim}

Hypothesis dependency analysis: \texttt{papers/umbrella-note/hypothesis\_matrix.py}

Falsification program: \texttt{papers/umbrella-note/falsification\_table.py}

\end{document}
