% Paper 1: Spandrel - SN Ia Population Evolution
% Does SN Ia Progenitor Evolution Mimic Evolving Dark Energy?
\documentclass[twocolumn,prd,showpacs,nofootinbib,superscriptaddress]{revtex4-2}

\usepackage{amsmath,amssymb}
\usepackage{graphicx}
\usepackage{hyperref}
\usepackage{booktabs}
\usepackage{xcolor}

% Draft notes
\newcommand{\todo}[1]{\textcolor{red}{[TODO: #1]}}
\newcommand{\note}[1]{\textcolor{blue}{[NOTE: #1]}}

\begin{document}

\title{Does SN Ia Progenitor Evolution Mimic Evolving Dark Energy?\\
A Host-Galaxy Diagnostic for DESI's $w_0$-$w_a$ Signal}

\author{COSMOS Collaboration}
\affiliation{TBD}

\date{\today}

\begin{abstract}
The DESI Year-1 results suggest evolving dark energy ($w_a \neq 0$) at
$2.5\sigma$--$3.9\sigma$ significance when combined with Type Ia supernova
data. We investigate whether this signal could arise from unmodeled
SN~Ia population evolution correlated with host galaxy properties---the
``Spandrel'' hypothesis. Using hierarchical Bayesian modeling, we test
whether high-mass and low-mass host galaxies yield consistent dark energy
parameters when population drift is (not) included. Under the Spandrel
hypothesis, baseline fits should show mass-dependent apparent cosmology,
while evolution-corrected fits should not. We present methodology and
simulated validation; application to Pantheon+ and DES-SN5YR samples
is ongoing. This differential diagnostic is robust to absolute calibration
uncertainties and provides a direct falsification test for the population
evolution scenario.
\end{abstract}

\maketitle

%%%%%%%%%%%%%%%%%%%%%%%%%%%%%%%%%%%%%%%%%%%%%%%%%%%%%%%%%%%%%%%%%%%%%%%%%%%%%%%
\section{Introduction}
\label{sec:intro}

\todo{Opening: DESI BAO + SN evidence for $w_a \neq 0$}

\todo{Problem: SN Ia standardization assumes population-invariant relations}

\todo{Spandrel hypothesis: If progenitor metallicity evolves with z, and
metallicity correlates with host mass, the host-mass split should reveal
systematic differences in apparent cosmology.}

\todo{This paper: Define the test, validate on simulations, apply to data}

Key references:
\begin{itemize}
    \item DESI Collaboration (2024, 2025): BAO and dark energy
    \item Scolnic et al. (2022): Pantheon+ sample
    \item Rigault et al. (2020): Host mass correlations
    \item Brout \& Scolnic (2021): Evolution systematics
\end{itemize}

%%%%%%%%%%%%%%%%%%%%%%%%%%%%%%%%%%%%%%%%%%%%%%%%%%%%%%%%%%%%%%%%%%%%%%%%%%%%%%%
\section{The Spandrel Hypothesis}
\label{sec:hypothesis}

\subsection{Physical Motivation}

SN Ia progenitor properties (C/O ratio, metallicity) may evolve with
lookback time. If this evolution affects peak luminosity beyond SALT2
standardization:
\begin{equation}
    M_B(z) = M_{B,0} + \frac{dM}{dz} z + \alpha x_1 - \beta c + \gamma \cdot \text{(host mass)}
\end{equation}

\todo{Discuss metallicity-mass relation in host galaxies}

\todo{Discuss progenitor delay-time distribution evolution}

\subsection{The Differential Signature}

If $dM/dz \neq 0$ correlates with host mass:
\begin{itemize}
    \item High-mass hosts: older stellar populations, higher metallicity
    \item Low-mass hosts: younger populations, lower metallicity
\end{itemize}

The falsification test: Split sample at $\log(M_*/M_\odot) = 10$ and fit
cosmology independently. Under Spandrel:
\begin{align}
    w_0^{\text{high}} &\neq w_0^{\text{low}} \quad \text{(baseline model)} \\
    w_0^{\text{high}} &= w_0^{\text{low}} \quad \text{(evolution model)}
\end{align}

%%%%%%%%%%%%%%%%%%%%%%%%%%%%%%%%%%%%%%%%%%%%%%%%%%%%%%%%%%%%%%%%%%%%%%%%%%%%%%%
\section{Hierarchical Bayesian Model}
\label{sec:model}

\subsection{Likelihood Function}

For $N$ SNe with observed magnitudes $\{m_i\}$:
\begin{equation}
    \ln \mathcal{L} = -\frac{1}{2} \sum_{i=1}^{N} \left[
        \frac{(m_i - \mu_i + M_{\text{std},i})^2}{\sigma_i^2}
        + \ln(2\pi\sigma_i^2)
    \right]
\end{equation}

where:
\begin{align}
    \mu_i &= \mu(z_i; H_0, \Omega_m, w_0, w_a) \\
    M_{\text{std},i} &= M_0 + \alpha x_{1,i} - \beta c_i + \gamma \max(\log M_* - 10, 0)
\end{align}

\subsection{Evolution Parameters}

The Spandrel model adds:
\begin{equation}
    M_{\text{std},i} \to M_{\text{std},i} + \frac{dM}{dz} z_i
\end{equation}

\todo{Full parameter vector: $\theta = \{M_0, \alpha, \beta, \gamma, \sigma_{\text{int}}, dM/dz, H_0, \Omega_m, w_0, w_a\}$}

\subsection{Priors}

\todo{Table of priors for all parameters}

\begin{table}[h]
\centering
\begin{tabular}{lcc}
\toprule
Parameter & Prior & Notes \\
\midrule
$M_0$ & $\mathcal{N}(-19.3, 0.5)$ & Absolute magnitude \\
$\alpha$ & $\mathcal{N}(0.14, 0.05)$ & Stretch coefficient \\
$\beta$ & $\mathcal{N}(3.1, 0.5)$ & Color coefficient \\
$\gamma$ & $\mathcal{N}(0.05, 0.05)$ & Host mass step \\
$\sigma_{\text{int}}$ & Half-$\mathcal{N}(0.1)$ & Intrinsic scatter \\
$dM/dz$ & $\mathcal{N}(0, 0.2)$ & Evolution (Spandrel) \\
$w_0$ & $\mathcal{U}(-2, 0)$ & DE equation of state \\
$w_a$ & $\mathcal{N}(0, 1)$ & DE evolution \\
\bottomrule
\end{tabular}
\caption{Prior distributions for model parameters.}
\label{tab:priors}
\end{table}

%%%%%%%%%%%%%%%%%%%%%%%%%%%%%%%%%%%%%%%%%%%%%%%%%%%%%%%%%%%%%%%%%%%%%%%%%%%%%%%
\section{Validation on Simulated Data}
\label{sec:validation}

\subsection{Injection-Recovery Tests}

\todo{Generate simulated SN samples with known $dM/dz$}

\todo{Verify baseline model shows host-mass dependent $w_0$, $w_a$}

\todo{Verify evolution model recovers true cosmology}

\subsection{Detection Sensitivity}

\todo{Minimum $dM/dz$ detectable with Pantheon+ statistics}

\todo{Expected significance vs sample size}

%%%%%%%%%%%%%%%%%%%%%%%%%%%%%%%%%%%%%%%%%%%%%%%%%%%%%%%%%%%%%%%%%%%%%%%%%%%%%%%
\section{Application to Real Data}
\label{sec:data}

\subsection{Pantheon+ Sample}

\todo{Sample selection and cuts}

\todo{Host mass distribution}

\subsection{Results: Host Mass Split}

\todo{Table: $w_0$, $w_a$ for high/low mass subsamples}

\todo{Figure: Posterior contours}

\subsection{Results: Model Comparison}

\todo{$\Delta$BIC between baseline and Spandrel models}

\todo{Bayes factor interpretation}

%%%%%%%%%%%%%%%%%%%%%%%%%%%%%%%%%%%%%%%%%%%%%%%%%%%%%%%%%%%%%%%%%%%%%%%%%%%%%%%
\section{Discussion}
\label{sec:discussion}

\subsection{Implications for DESI Dark Energy Results}

\todo{If Spandrel confirmed: DESI $w_a$ signal may be systematic}

\todo{If Spandrel rejected: Strengthens case for true DE evolution}

\subsection{Comparison to Other Systematics Tests}

\todo{Relation to existing host mass step analyses}

\todo{Difference: We test for z-dependent differential, not constant step}

\subsection{Future Prospects}

\todo{LSST, Roman: Much larger samples with better host characterization}

\todo{Spectroscopic host metallicities as direct probe}

%%%%%%%%%%%%%%%%%%%%%%%%%%%%%%%%%%%%%%%%%%%%%%%%%%%%%%%%%%%%%%%%%%%%%%%%%%%%%%%
\section{Conclusions}
\label{sec:conclusions}

\begin{enumerate}
    \item We propose a differential diagnostic for SN Ia population evolution
    \item The host-mass split test is robust to absolute calibration
    \item Simulations validate the methodology
    \item Application to Pantheon+ is [pending/complete]
    \item Result: [Spandrel confirmed/rejected at X$\sigma$]
\end{enumerate}

%%%%%%%%%%%%%%%%%%%%%%%%%%%%%%%%%%%%%%%%%%%%%%%%%%%%%%%%%%%%%%%%%%%%%%%%%%%%%%%
\section*{Acknowledgments}

\todo{Acknowledgments}

%%%%%%%%%%%%%%%%%%%%%%%%%%%%%%%%%%%%%%%%%%%%%%%%%%%%%%%%%%%%%%%%%%%%%%%%%%%%%%%
\appendix

\section{Distance Modulus Calculation}
\label{app:distance}

For flat $w_0$-$w_a$ cosmology:
\begin{equation}
    \mu(z) = 5\log_{10}\left(\frac{d_L(z)}{\text{Mpc}}\right) + 25
\end{equation}

where the luminosity distance:
\begin{equation}
    d_L(z) = \frac{c(1+z)}{H_0} \int_0^z \frac{dz'}{E(z')}
\end{equation}

and:
\begin{equation}
    E(z) = \sqrt{\Omega_m(1+z)^3 + \Omega_{\text{DE}}(1+z)^{3(1+w_0+w_a)} e^{-3w_a z/(1+z)}}
\end{equation}

\section{MCMC Implementation Details}
\label{app:mcmc}

\todo{Sampler settings: emcee, n\_walkers, n\_steps, burn-in}

\todo{Convergence diagnostics: Gelman-Rubin, autocorrelation}

%%%%%%%%%%%%%%%%%%%%%%%%%%%%%%%%%%%%%%%%%%%%%%%%%%%%%%%%%%%%%%%%%%%%%%%%%%%%%%%
\bibliography{spandrel_refs}

\end{document}
