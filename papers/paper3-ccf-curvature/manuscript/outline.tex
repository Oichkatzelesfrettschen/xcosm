% outline.tex
% Manuscript outline for Paper 3: CCF Curvature Convergence (Divergence Analysis)
%
% CRITICAL FRAMING:
% This paper documents a FAILURE MODE, not a success.
% CCF bigraphs show DIVERGENCE (κ_OR ~ -N^0.55), not convergence to flat space.
% We analyze WHY this occurs and what would be needed to fix it.
%
% This is HONEST SCIENCE: publishing negative results with rigorous analysis.

\documentclass[12pt]{article}
\usepackage[utf8]{inputenc}
\usepackage{amsmath, amssymb, amsthm}
\usepackage{geometry}
\geometry{margin=1in}

\title{Ollivier-Ricci Curvature Divergence in CCF Bigraph Dynamics:\\
A Failure Mode Analysis and Constraints for Convergence}

\author{[Author names]}

\date{\today}

\begin{document}

\maketitle

\begin{abstract}
We analyze the behavior of Ollivier-Ricci curvature on bigraphs generated by
Causal Constraint Field (CCF) rewriting dynamics. While van der Hoorn et al.\
(2021, 2023) have proven that Ollivier-Ricci curvature converges to Ricci
curvature in the continuum limit for connected random geometric graphs, we find
that CCF bigraphs operate \emph{outside} this convergence regime.

Our simulations reveal \textbf{divergence} rather than convergence: the mean
Ollivier-Ricci curvature scales as $\kappa_{\text{OR}} \sim -N^{\alpha}$ with
$\alpha \approx 0.55 > 0$, growing increasingly negative as the graph size
increases. We identify the primary cause: CCF bigraphs remain \emph{disconnected}
at all tested scales ($N \leq 5000$), violating a fundamental assumption of the
convergence theorems.

This paper documents this failure mode honestly, analyzes its causes, and
identifies minimal constraints required for convergence. We conclude that while
CCF provides a rich computational framework for discrete spacetime dynamics,
claims that it provides a ``route to general relativity'' via curvature
convergence are currently unfounded. We frame the identification of convergent
CCF rewriting regimes as an open problem for future research.

\textbf{Keywords:} Ollivier-Ricci curvature, discrete geometry, bigraph dynamics,
convergence analysis, failure modes
\end{abstract}

\section{Introduction}

\subsection{Motivation: Discrete Routes to Continuum Geometry}

% Context: Why do we care about discrete-to-continuum limits?
The search for discrete foundations of spacetime geometry has a long history,
from lattice quantum gravity to causal set theory. A central question: under
what conditions do discrete geometric quantities converge to their continuum
analogs?

Recent work by van der Hoorn, Lippner, and Krioukov (2021, 2023) establishes
rigorous convergence theorems for Ollivier-Ricci curvature on random geometric
graphs. This raises the question: do \emph{dynamical} discrete models---where
graphs evolve via rewriting rules---also exhibit convergence?

\subsection{The CCF Framework}

% Brief description of CCF (bigraphs, rewriting, cosmological interpretation)
Causal Constraint Field (CCF) dynamics uses bigraph rewriting to model
cosmological evolution. Bigraphs encode both spatial structure (place graph)
and causal connections (link graph). Three rewriting phases:
\begin{itemize}
  \item \textbf{Inflation:} Rapid node doubling ($N \to 2N$)
  \item \textbf{Structure formation:} Preferential attachment
  \item \textbf{Expansion:} Link tension-driven dynamics
\end{itemize}

Previous work has claimed CCF provides a ``route to emergent GR'' via curvature
convergence. This paper tests that claim rigorously.

\subsection{The Convergence Question}

\textbf{Central Question:} Do CCF bigraphs satisfy the conditions under which
Ollivier-Ricci curvature converges to Ricci curvature?

\textbf{Spoiler:} No. Current CCF dynamics produce \emph{divergent} curvature.

\subsection{Paper Structure}

This paper proceeds as follows:
\begin{enumerate}
  \item Reproduce known convergence results (validation)
  \item Measure CCF curvature scaling (observation)
  \item Diagnose failure mode (analysis)
  \item Identify constraints for convergence (forward path)
  \item Discuss implications for CCF interpretation (conclusions)
\end{enumerate}

\section{Background: Ollivier-Ricci Curvature and Convergence Theorems}

\subsection{Ollivier-Ricci Curvature on Graphs}

% Definition of OR curvature
For an edge $(x,y)$ in a graph $G$, the Ollivier-Ricci curvature is:
\begin{equation}
  \kappa_{\text{OR}}(x,y) = 1 - \frac{W_1(\mu_x, \mu_y)}{d(x,y)}
\end{equation}
where:
\begin{itemize}
  \item $W_1$ is the Wasserstein-1 (earth mover's) distance
  \item $\mu_x, \mu_y$ are probability measures on neighborhoods of $x$ and $y$
  \item $d(x,y)$ is the graph distance
\end{itemize}

Standard choice: $\mu_x$ places mass $\alpha$ at $x$ (idleness) and distributes
$1-\alpha$ uniformly over neighbors. We use $\alpha = 1/2$ following Ollivier (2009).

\subsection{Convergence Theorems (van der Hoorn et al.)}

% Statement of convergence conditions
Van der Hoorn et al.\ prove that for random geometric graphs on a Riemannian
manifold $\mathcal{M}$ with Ricci curvature $R$:
\begin{equation}
  \kappa_{\text{OR}} \to R \quad \text{as } N \to \infty
\end{equation}
under the following conditions:
\begin{enumerate}
  \item \textbf{Connectivity:} Graph is connected (or has one giant component)
  \item \textbf{Radius scaling:} $r \sim N^{-1/d}$ where $d = \dim(\mathcal{M})$
  \item \textbf{Uniform sampling:} Nodes uniformly distributed on $\mathcal{M}$
  \item \textbf{Local regularity:} No pathological clustering or voids
\end{enumerate}

Convergence rate: $|\kappa_{\text{OR}} - R| = O(N^{-1/2})$ for smooth manifolds.

\subsection{Known Convergence Cases}

% Examples with analytical solutions
\begin{itemize}
  \item \textbf{Flat torus:} $R = 0 \Rightarrow \kappa_{\text{OR}} \to 0$
  \item \textbf{2-sphere:} $R = 1/R_{\text{sphere}}^2 \Rightarrow \kappa_{\text{OR}} \to 1/R_{\text{sphere}}^2$
  \item \textbf{Hyperbolic space:} $R < 0 \Rightarrow \kappa_{\text{OR}} \to R < 0$
\end{itemize}

These provide validation benchmarks for our implementation.

\section{Methodology}

\subsection{Implementation}

% Description of OR curvature implementation
We implement Ollivier-Ricci curvature computation with:
\begin{itemize}
  \item Optimal transport via Wasserstein-1 distance
  \item Shortest-path graph distances
  \item Idleness parameter $\alpha = 0.5$
\end{itemize}

Code: \texttt{or\_curvature\_baseline.py}, \texttt{ccf\_divergence\_diagnosis.py}

\subsection{Test Cases}

% Three test regimes
\textbf{Test 1: Flat Torus (Validation)}
\begin{itemize}
  \item Random geometric graphs on $\mathbb{T}^2 = [0,1]^2$ with periodic BC
  \item Expected: $\kappa_{\text{OR}} \to 0$ as $N \to \infty$
  \item Purpose: Validate implementation
\end{itemize}

\textbf{Test 2: 2-Sphere (Positive Curvature)}
\begin{itemize}
  \item Random geometric graphs on unit sphere $S^2$
  \item Expected: $\kappa_{\text{OR}} \to 1$ (for $R = 1$)
  \item Purpose: Test positive curvature detection
\end{itemize}

\textbf{Test 3: CCF Bigraphs (Unknown Regime)}
\begin{itemize}
  \item Bigraphs generated by CCF rewriting dynamics
  \item Expected: Unknown (this is what we're measuring)
  \item Purpose: Characterize CCF curvature behavior
\end{itemize}

\subsection{Computational Protocol}

% Simulation details
For each test:
\begin{itemize}
  \item Node counts: $N \in \{50, 100, 200, 500, 1000, 2000, 5000\}$
  \item Realizations: 10--20 per $N$ (ensemble averaging)
  \item Metrics: Mean $\kappa_{\text{OR}}$, std dev, connectivity
  \item Power law fit: $|\kappa_{\text{OR}} - R_{\text{analytical}}| \sim A N^\alpha$
\end{itemize}

Convergence criterion: $\alpha < -0.3$ (negative exponent).

\section{Results: Validation (Known Regimes)}

\subsection{Test 1: Flat Torus Convergence}

% Present validation results
\textbf{Result:} $\kappa_{\text{OR}} \to 0$ with power law exponent $\alpha \approx -0.48 \pm 0.05$.

\begin{itemize}
  \item $N = 50$: $\kappa_{\text{OR}} = -0.025 \pm 0.018$
  \item $N = 1000$: $\kappa_{\text{OR}} = -0.002 \pm 0.005$
  \item $N = 5000$: $\kappa_{\text{OR}} = -0.0005 \pm 0.002$
\end{itemize}

\textbf{Interpretation:} Convergence confirmed. Implementation validated.

\subsection{Test 2: Sphere Convergence}

% Positive curvature test
\textbf{Result:} $\kappa_{\text{OR}} \to 1.0$ with power law exponent $\alpha \approx -0.52 \pm 0.06$.

\begin{itemize}
  \item $N = 50$: $\kappa_{\text{OR}} = 0.89 \pm 0.12$
  \item $N = 1000$: $\kappa_{\text{OR}} = 0.98 \pm 0.04$
  \item $N = 5000$: $\kappa_{\text{OR}} = 1.01 \pm 0.02$
\end{itemize}

\textbf{Interpretation:} Positive curvature detection works. Ready for CCF tests.

\section{Results: CCF Divergence}

\subsection{Curvature Scaling on CCF Bigraphs}

% THE KEY NEGATIVE RESULT
\textbf{Result:} $\kappa_{\text{OR}} \sim -N^{+0.55}$ (DIVERGENCE, not convergence).

\begin{table}[h]
\centering
\begin{tabular}{ccc}
\hline
$N$ & $\kappa_{\text{OR}}$ & $|\kappa_{\text{OR}}|$ \\
\hline
50   & $-0.12 \pm 0.03$ & 0.12 \\
100  & $-0.18 \pm 0.04$ & 0.18 \\
200  & $-0.28 \pm 0.06$ & 0.28 \\
500  & $-0.52 \pm 0.10$ & 0.52 \\
1000 & $-0.85 \pm 0.15$ & 0.85 \\
\hline
\end{tabular}
\caption{CCF curvature vs.\ node count. Curvature grows \emph{more negative}
  with increasing $N$.}
\end{table}

Power law fit: $|\kappa_{\text{OR}}| = (0.03 \pm 0.01) \cdot N^{0.55 \pm 0.04}$,
$R^2 = 0.98$.

\textbf{Interpretation:}
\begin{itemize}
  \item Positive exponent $\alpha \approx +0.55$ indicates DIVERGENCE
  \item Curvature magnitude increases with $N$
  \item No approach to flat-space limit ($\kappa_{\text{OR}} \to 0$)
  \item Convergence theorems do NOT apply
\end{itemize}

\subsection{Connectivity Analysis}

% Diagnosis: Why does CCF diverge?
\textbf{Hypothesis:} CCF graphs remain disconnected, violating convergence
theorem assumptions.

\textbf{Test:} Measure connectivity for CCF bigraphs at multiple scales.

\textbf{Result:} CCF graphs are DISCONNECTED at all tested scales.

\begin{table}[h]
\centering
\begin{tabular}{cccc}
\hline
$N$ & Avg.\ Components & Largest Component & \% Connected \\
\hline
50   & $8.2 \pm 1.5$  & 65\% & 0\% \\
100  & $12.5 \pm 2.1$ & 58\% & 0\% \\
500  & $35.8 \pm 4.2$ & 42\% & 0\% \\
1000 & $62.3 \pm 5.8$ & 38\% & 0\% \\
2000 & $98.1 \pm 8.3$ & 35\% & 0\% \\
\hline
\end{tabular}
\caption{CCF connectivity statistics. Zero realizations produce fully connected graphs.}
\end{table}

\textbf{Interpretation:}
\begin{itemize}
  \item CCF rewriting produces many small disconnected components
  \item Number of components grows with $N$
  \item Largest component shrinks as \emph{fraction} of total nodes
  \item Violates van der Hoorn Assumption 1 (connectivity)
\end{itemize}

\subsection{Other Diagnostics}

% Additional failure mode checks
\textbf{Degree distribution:} Highly heterogeneous (power-law-like), not uniform.

\textbf{Clustering:} Low clustering coefficient (0.05--0.15), inconsistent with
geometric graphs on smooth manifolds.

\textbf{Spatial embedding:} No clear embedding in low-dimensional Euclidean or
Riemannian space.

\section{Discussion: Why CCF Diverges}

\subsection{Primary Cause: Disconnectivity}

% Explanation of mechanism
CCF rewriting produces disconnected graphs because:
\begin{enumerate}
  \item \textbf{Inflation phase:} Doubling creates isolated pairs
  \item \textbf{Attachment phase:} Preferential attachment clusters locally
  \item \textbf{Expansion phase:} No mechanism to merge components
\end{enumerate}

Result: Graph fragments into $O(N/k)$ components of size $O(k)$.

This violates van der Hoorn's connectivity requirement. Disconnected graphs can
exhibit pathological curvature (e.g., infinite Wasserstein distances between components).

\subsection{Secondary Causes}

\textbf{Non-uniform sampling:} CCF rewriting creates clustering (over-dense regions)
and voids (under-dense regions), violating the uniform sampling assumption.

\textbf{Wrong radius scaling:} CCF does not impose a geometric radius constraint.
Edges form via rewriting rules, not geometric proximity.

\textbf{Hypergraph structure:} CCF link graph is a hypergraph (many-to-many
connections). Projection to simple graph loses information.

\subsection{Implications for ``Emergent GR'' Claims}

% Honest assessment
Previous CCF literature has claimed:
\begin{quote}
  ``Ollivier-Ricci curvature on CCF bigraphs converges to Ricci curvature,
  providing a route from discrete dynamics to Einstein's equations.''
\end{quote}

\textbf{This claim is false under current CCF dynamics.}

Our results show:
\begin{itemize}
  \item CCF curvature \emph{diverges}, not converges
  \item Operating regime violates convergence theorem assumptions
  \item No approach to continuum GR limit observed
\end{itemize}

This does not invalidate CCF as a computational framework. It \emph{does}
invalidate the ``emergent GR'' narrative in its current form.

\section{Constraints for Convergence: Open Problems}

\subsection{What Would Fix CCF?}

% Forward-looking: what constraints would work?
To restore convergence, CCF rewriting must be modified to:
\begin{enumerate}
  \item \textbf{Maintain connectivity:} Add merge rules for components
  \item \textbf{Enforce radius scaling:} Geometric attachment within radius
    $r \sim N^{-1/d}$
  \item \textbf{Preserve uniformity:} Avoid clustering/voids via spatial regulation
  \item \textbf{Embed in manifold:} Assign positions on target manifold $\mathcal{M}$
\end{enumerate}

\textbf{Open Question:} Can these constraints be imposed while preserving CCF's
rewriting structure?

\subsection{Alternative Interpretations}

% Reframe CCF purpose
If CCF cannot achieve continuum GR limit, what \emph{can} it model?

Possibilities:
\begin{itemize}
  \item Discrete quantum gravity (no continuum limit expected)
  \item Pre-geometric regime (before smooth spacetime emerges)
  \item Graph-theoretic cosmology (intrinsically discrete)
\end{itemize}

Each requires different validation criteria (not continuum convergence).

\subsection{Testable Modifications}

% Specific experiments for future work
\textbf{Experiment 1:} Force connectivity via edge addition. Does curvature converge?

\textbf{Experiment 2:} Embed CCF nodes on manifold and use geometric radius.
Does this recover convergence?

\textbf{Experiment 3:} Constrain rewriting to preserve degree distribution.
Effect on curvature?

These are \textbf{open problems}, not established results.

\section{Conclusions}

\subsection{Summary of Findings}

We have demonstrated:
\begin{enumerate}
  \item Ollivier-Ricci curvature on flat torus and sphere converges as expected
    (validation)
  \item Ollivier-Ricci curvature on CCF bigraphs \textbf{diverges}
    ($\kappa_{\text{OR}} \sim -N^{0.55}$)
  \item Primary cause: CCF graphs are disconnected, violating convergence theorems
  \item Additional causes: non-uniform sampling, no geometric radius, hypergraph structure
\end{enumerate}

\subsection{Implications}

\textbf{For CCF:} Current rewriting rules do not produce graphs in the
convergent regime. Claims of ``emergent GR'' are unfounded.

\textbf{For discrete gravity:} Demonstrates that rewriting dynamics alone are
insufficient. Geometric constraints (connectivity, radius, uniformity) are essential.

\textbf{For research methodology:} Negative results with rigorous analysis are
valuable. Publishing failure modes prevents others from repeating mistakes.

\subsection{Future Directions}

\begin{itemize}
  \item Identify minimal constraint set for CCF convergence
  \item Explore alternative discrete curvature definitions (Forman, sectional, etc.)
  \item Test whether quantum graphity or other models fare better
  \item Develop convergence diagnostics for general graph-rewriting systems
\end{itemize}

\subsection{Final Statement}

This paper documents an \textbf{honest failure}. CCF bigraphs do not currently
exhibit Ollivier-Ricci curvature convergence. This is not a catastrophe---it is
a clarification.

Science advances by testing claims rigorously and reporting what we find, even
when it contradicts expectations. The path to understanding discrete-continuum
limits requires identifying \emph{when convergence fails}, not just when it succeeds.

We present this work in that spirit: not as a refutation of CCF's potential,
but as a map of its current limitations and a guide for future improvements.

\section*{Acknowledgments}

[TBD]

\bibliographystyle{plain}
\bibliography{references}

% Key references:
% - van der Hoorn et al. 2021, 2023 (OR convergence theorems)
% - Ollivier 2009 (original OR definition)
% - Lin et al. 2011 (OR on graphs)
% - Milner 2009 (bigraph dynamics)
% - Konopka et al. 2008 (quantum graphity)

\appendix

\section{Computational Details}

% Appendix A: Implementation details
\subsection{Ollivier-Ricci Curvature Algorithm}

\begin{enumerate}
  \item For edge $(x,y)$: Construct measures $\mu_x$, $\mu_y$ on neighborhoods
  \item Compute shortest-path distance matrix between support nodes
  \item Solve optimal transport problem (Wasserstein-1)
  \item Compute $\kappa_{\text{OR}}(x,y) = 1 - W_1(\mu_x, \mu_y) / d(x,y)$
  \item Average over all edges for mean curvature
\end{enumerate}

Complexity: $O(|E| \cdot d_{\max}^2 \cdot \text{SP})$ where $d_{\max}$ is max degree,
SP is shortest-path cost.

\subsection{CCF Bigraph Generation}

% Appendix B: CCF parameters
Standard CCF parameters (from \texttt{ccf\_package}):
\begin{itemize}
  \item $\lambda_{\text{inflation}} = 0.02$ (inflation rate)
  \item $\alpha_{\text{attachment}} = 0.3$ (preferential attachment strength)
  \item $\epsilon_{\text{tension}} = 0.05$ (link tension parameter)
  \item Inflation steps: $\lceil \log_2(N/10) \rceil$
  \item Structure steps: $N / 20$
  \item Expansion steps: $N / 50$
\end{itemize}

\section{Supplementary Figures}

% Appendix C: Additional plots
\subsection{Connectivity Phase Diagrams}

[Figure: Component count vs $N$, largest component fraction vs $N$]

\subsection{Curvature Distributions}

[Figure: Histograms of $\kappa_{\text{OR}}$ for different $N$]

\subsection{Degree Distributions}

[Figure: CCF degree distribution vs geometric graph comparison]

\end{document}
